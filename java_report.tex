\section{Java 学习报告}

 Java 是一门面向对象的程序设计语言。 外观看上去和C++很像。 但实际上Java是完全OOP风格的,
 就连main入口函数都要包在某个类里面, 可见其设计者的洁癖程度。

 我基本没有接触过Java,不过在某次做游戏时学习了几个标准类(String, Array, StringBuilder, Scanner, OutputStreamWriter, Socket)的用法, 
 所以谈谈一个初学者的粗浅见解。
 
 在写了一个HelloWorld之后, 发现Java的类声明中每个成员(变量或函数)都要声明其访问属性,这样有利于灵活性和代码但维护。
 另外Java的一个类只在一个文件里面, 省去了在头文件声明一次,又在实现文件里面写一次函数头但麻烦。

 到目前为止, 我最喜欢Java的一点是,其所有变量都是类实例的引用(为了提高速度,数值类型可能例外做了优化)。而且内存是自动
 管理回收的,让程序员可以把注意力集中到更关键的地方, 不用像写C程序一样时刻担心内存管理问题. 另外java是运行在JVM(Java虚拟机)上的,有点像
 动态语言但解析器,这样Java但可移植性大大提高。这几点使Java在使用上和一些动态语言比较相似(例如 Python)。

 下面是我配置Java环境的过程:

 \begin{itemize}
     \item 由于我使用的OS是Archlinux, 所以我用pacman(包管理器)来安装java环境。
     \item 根据Sun官方网站的安装指导, 知道需要JRE和JDK这两样东西
     \item 用命令:
         \begin{verbatim}
         # pacman -Ss jdk
         \end{verbatim}
         搜索得到arch官方源里面有jdk7-openjdk和jre7-openjdk可以用。
     \item 用命令
         \begin{verbatim}
         # pacman -S jdk7-openjdk jre7-openjdk
         \end{verbatim}
         进行安装。
 \end{itemize}
当前两者的版本都是 7.u5 .

 至于环境变量的配置, 安装完重启session发现环境变量中已经有 JAVA\_Home了, 于是用\verb=pacman -Ql jdk7-openjdk|grep etc=命令查看了这个包但配置文件,
 发现在 /etc/profile.d/ 里面有配置文件 jdk.sh, jre.sh。已经完成了配置。

 \begin{verbatim}
 # configs in jdk.sh
 export J2SDKDIR=/usr/lib/jvm/java-7-openjdk
 export J2REDIR=$J2SDKDIR/jre
 export JAVA_HOME=/usr/lib/jvm/java-7-openjdk

 # configs in jre.sh
 export J2REDIR=/usr/lib/jvm/java-7-openjdk/jre
 export JAVA_HOME=${JAVA_HOME:-/usr/lib/jvm/java-7-openjdk/jre}
 \end{verbatim}

 然后写了一个HelloWorld.java来测试, 执行:
 \begin{verbatim}
 # vim HelloWorld.java
  ...
 # javac HelloWorld.java && java HelloWorld.class
 \end{verbatim}
 成功输出。
